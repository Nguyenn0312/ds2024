\documentclass{article}


\usepackage[english]{babel}
\usepackage{listings}

\usepackage[letterpaper,top=2cm,bottom=2cm,left=3cm,right=3cm,marginparwidth=1.75cm]{geometry}

% Useful packages
\usepackage{amsmath}
\usepackage{graphicx}
\usepackage[colorlinks=true, allcolors=blue]{hyperref}

\title{Lab01 \\ Implement File transfer protocol }
\author{BI12-328 Nguyen Le Nguyen}

\begin{document}
\maketitle

\begin{abstract}
Your abstract.
\end{abstract}

\section{Introduction}

File Transfer Protocol (FTP) is a standard network protocol used to transfer files between a client and a server on a computer network. In this report, we present the implementation of a basic FTP using the TCP protocol in the C programming language. 

This implementation includes functionality for the server to hold files and open a listening socket, and for the client to establish a connection to the server, receive file data, and save the file locally.

\section{Implementation:}

\subsection{Server-Side Implementation:}

\begin{itemize}
  \item The server program is responsible for holding the files and accepting connections from clients.
  \item It creates a socket using the socket() system call and binds it to a specific port using the bind() system call.
  \item The server listens for incoming connections from clients using the listen() system call.
  \item When a client connects, the server accepts the connection using the accept() system call and spawns a new thread or process to handle the client's request.
  \item Upon receiving a request from the client, the server sends the requested file to the client over the established TCP connection.
\end{itemize}

\subsection{Client-side Implementation}

\begin{itemize}
\item The client program establishes a connection to the server using the socket() and connect() system calls.
\item Once connected, the client sends a request to the server to transfer a specific file.
\item Upon receiving the file data from the server, the client saves the data to a local file using standard file I/O functions such as fopen() and fwrite().
\end{itemize}

\subsection{File transfer implement }

Implementation of File Transfer

The file transfer is implemented using socket programming in C. The server and client communicate over TCP/IP sockets using the socket(), bind(), listen(), accept(), connect(), send(), and recv() system calls.

\section{Code Snippets}
Server code:
\begin{lstlisting}
#include <stdio.h>
#include <stdlib.h>
#include <string.h>
#include <arpa/inet.h>

#define SIZE 1024

void write_file(int sockfd)
{
    int n; 
    FILE *fp;
    char *filename = "file.txt";
    char buffer[SIZE];

    fp = fopen(filename, "w");
    if(fp==NULL)
    {
        perror("Error in creating file.");
        exit(1);
    }
    while(1)
    {
        n = recv(sockfd, buffer, SIZE, 0);
        if(n<=0)
        {
            break;
            return;
        }
        fprintf(fp, "%s", buffer);
        bzero(buffer, SIZE);
    }
    return;
    
}

int main ()
{
    char *ip = "127.0.0.1";
    int port = 8080;
    int e;

    int sockfd, new_sock;
    struct sockaddr_in server_addr, new_addr;
    socklen_t addr_size;
    char buffer[SIZE];

    sockfd = socket(AF_INET, SOCK_STREAM, 0);
    if(sockfd<0)
    {
        perror("Error in socket");
        exit(1);
    }
     printf("Server socket created. \n");

     server_addr.sin_family = AF_INET;
     server_addr.sin_port = port;
     server_addr.sin_addr.s_addr = inet_addr(ip);

     e = bind(sockfd,(struct sockaddr*)&server_addr, sizeof(server_addr));
     if(e<0)
     {
         perror("Error in Binding");
         exit(1);
     }
     printf("Binding Successfull.\n");

     e = listen(sockfd, 10);
     if(e==0)
     {
         printf("Listening...\n");
     }
     else 
     {
         perror("Error in Binding");
         exit(1);
     }
     addr_size = sizeof(new_addr);
     new_sock = accept(sockfd,(struct sockaddr*)&new_addr, &addr_size);

     write_file(new_sock);
     printf("Data written in the text file ");
}

\end{lstlisting}

Clent code:
\begin{lstlisting}
#include <stdio.h>
#include <stdlib.h>
#include <unistd.h>
#include <string.h>
#include <arpa/inet.h>

#define SIZE 1024

void send_file(FILE *fp, int sockfd)
{
    char data[SIZE] = {0};

    while(fgets(data, SIZE, fp)!=NULL)
    {
        if(send(sockfd, data, strlen(data), 0)== -1) // Change sizeof(data) to strlen(data)
        {
            perror("Error in sending data");
            exit(1);
        }
        bzero(data, SIZE);
    }
}

int main()
{
    char *ip = "127.0.0.1";
    int port = 8080;
    int e;

    int sockfd;
    struct sockaddr_in server_addr;
    FILE *fp;
    char *filename = "file.txt";
     sockfd = socket(AF_INET, SOCK_STREAM, 0);
    if(sockfd<0)
    {
        perror("Error in socket");
        exit(1);
    }
     printf("Server socket created. \n");

     server_addr.sin_family = AF_INET;
     server_addr.sin_port = htons(port); // Change port to network byte order using htons
     server_addr.sin_addr.s_addr = inet_addr(ip);

     e = connect(sockfd, (struct sockaddr*)&server_addr, sizeof(server_addr));
     if(e == -1)
     {
         perror("Error in Connecting");
         exit(1);
     }
     printf("[+]Connected to server.\n");
     fp = fopen(filename, "r");
     if(fp == NULL)
     {
         perror("Error in reading file.");
         exit(1);
     }
     send_file(fp,sockfd);
     printf(" File data sent successfully. \n"); // Change send to sent
     close(sockfd);
     printf("Disconnected from the server. \n");
     return 0;

}

\end{lstlisting}

File.txt:\\
\begin{lstlisting}
 Hi This is Nguyen Le Nguyen known as BI12-328
\end{lstlisting}

\subsection{Output}
Client:\\
Server socket created.\\
Connected to server.\\
File data sent successfully. \\
Disconnected from the server. \\
Server:\\
Server socket created. \\
Binding Successfull.\\
Listening...\\
Data written in the text file \\                                                                       

\end{document}